\documentclass[a4paper,twoside=false,abstract=false,numbers=noenddot,
titlepage=false,headings=small,parskip=half,version=last]{scrartcl}
\usepackage[utf8]{inputenc}
\usepackage[T1]{fontenc}
\usepackage[english]{babel}
\usepackage[colorlinks=true, pdfstartview=FitV,
linkcolor=black, citecolor=black, urlcolor=blue]{hyperref}
\usepackage{verbatim}
\usepackage{graphicx}
\usepackage{multirow}

\usepackage{tikz}
\usetikzlibrary{matrix}

\usepackage{amsmath}
\usepackage{amsthm}
\usepackage{amssymb}
\usepackage{amsfonts}

\usepackage{float}

\usepackage{gensymb}

\usepackage{authblk}

\usepackage{helpers}


\title{Solid State Physics - IM2601}
\subtitle{Laboration 3}
    \author[1]{Fredrik Forsberg}
    \author[1]{Jim Holmström}
    \author[1]{Samuel Zackrisson}
    \affil[1]{Engineering Physics, Royal Institute of Technology}
    \affil[1]{\{fforsber, jimho, samuelz\}@kth.se}


\begin{document}
\maketitle
\thispagestyle{empty}


In order to harness energy from light, a variant of the diode can be used. The photoviltaic cell, or solar cell, is a large verison of the p-n junction, where a thin net of conducting material which don't obscure the sunlight is placed on top of the p- and n-doped semiconductors, who in turn are connected to a matal surface. This result in a large p-n junction. On the n-side of the junction there are atom which can donate electrons while there are acceptor atoms on the p-side. This results in a permanent electric field over the junction which can stop the flow of electrons if the voltage is too low.

When the energy of the light hitting the solar cell is greater than the band gap of the diode, $E_g$, electrons will be excited and a current will start to flow in the direction opposite the electric field.



In our laboration we examine a thin film of polycristalline Silicon (Si). At a temperature of $T = 300 K$ the band gap of a Silicon diode is $E_g=1.11 eV$. The minimum frequency of light which will be able to create a current is given by

$$\nu = \frac{E_g}{h} \approx 2.689*10^{14} Hz$$

where $h$ is the Planck constant.


We use a Power Cassy\textsuperscript{TM} in order to measure the current under applied bias and during illumination by a standard desk lamp.
When no bias is applied 
\plot{distancedepence}{Generated current over distance from light source with
zero bias (blue) and generated current for only ambient light (red).}

\section{Discussion of the results}


\section{Conclusion}
Stuff happened, y0.

\begin{thebibliography}{1}
    \bibitem{Kittel}
        Charles Kittel,
        {\em Introduction to Solid State Physics 8th Edition},
        2005.
\end{thebibliography}


\end{document}
